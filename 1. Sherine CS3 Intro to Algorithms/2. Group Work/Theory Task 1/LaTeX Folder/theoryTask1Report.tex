\documentclass[]{article}
\usepackage{geometry}
\geometry{letterpaper}


% Title Page
\title{CS3 Theory Task 1}
\author{Anthony Cade Streich}


\begin{document}
\maketitle
\section*{Question 1.15}
Give a sequence of input pairs that causes the weighted quick-union algorithm with path compression by halving to produce a path of length 4.
\newline
\newline
\newline
Begin with a sequence of unique integers from the range of [0,7]. Execute the following union commands to build an algorithm with a path length of 4:

\begin{itemize}
	\item {1. union(0,1);}
	\item {2. union(2,3);}
	\item {3. union(4,5);}
	\item {4. union(6,7);}
	
	\item {5. union(1,2); // This union joins sequences (0,1) and (2,3).}
	\item {6. union(5,6); // This union joins sequences (4,5) and (6,7).}
	\item {7. union(0,4); // This union joins the two trees by unionizing (0,1)-(2,3) on "top" of (4,5)-(6,7), resulting in a path length of 4.}
\end{itemize}

\pagebreak

\section*{Question 1.22}
Modify Program 1.4 (Path Compression by Halving) to generate random pairs of integers between 0 and N-1 instead of reading them from standard input, and to loop until N-1 union operations have been performed. Run your program for N = $10^{3}$, $10^{4}$, $10^{5}$, and $10^{6}$ and print out the total number of edges generated for each value of N.

\section*{Code}
\#include <iostream>\newline
\#include <random>\newline
\newline
int main(int argc, char *argv[ ])\newline
\{\newline
\indent std::default_random_engine randomNumGenerator;\newline
	

\newline \}



\end{document}          
